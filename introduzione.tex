\chapter*{Introduzione}
\label{ch:introduzione}

Quando si tratta l'equazione di Schr\"odinger, 
\begin{equation}
    \centering
    i \hbar \frac{\partial}{\partial t} \psi (\vec{x}, t) = \hat{H} (\vec{x}, t) \psi (\vec{x}, t) \; \text{,}
    \label{eq:Sc}
\end{equation}
in meccanica quantistica il numero di problemi che si possono risolvere in maniera esatta è piuttosto limitato. Per fare previsioni e ottenere risultati concreti, è quindi di fondamentale importanza utilizzare metodi di soluzione numerici.

Si propone un metodo per la soluzione numerica dell'eq. (\ref{eq:Sc}) che consiste nell'applicazione ripetuta dell'espressione infinitesima dell'operatore di evoluzione temporale, approssimato tramite l'espansione di  Lie-Trotter.

È stato sviluppato un software che consiste di tre componenti principali: 
\begin{itemize} [nolistsep, leftmargin=1.5cm]
    \item un ambiente grafico per poter gestire i parametri della simulazione,
    \item l'algoritmo simulativo,
    \item una serie di strumenti grafici per mostrare i risultati tramite animazioni.
\end{itemize}
Si è utilizzato come linguaggio di programmazione Python, ma parte dell'algoritmo è stato implementato in modo che fosse comapatibile con il compilatore Numba, questo per ridurre i tempi di simulazione. 

Il metodo è stato innanzitutto validato tramite il confronto delle simulazioni con alcune soluzione esatte, tra cui la dinamica degli stati coerenti dell'oscillatore armonico e, nel potenziale di P\"oschl-Teller, l'evoluzione di un pacchetto gaussiano di autostati. In seguito, si è utilizzato il software per risolvere alcuni dei problemi monodimensionali generalmente proposti nei corsi di meccanica quantistica e altri di particolare interesse.

\section{L'equazione di Schr\"odinger}
\label{sec:idea}

La più generica forma dell'equazione di Schr\"odinger risulta essere quella in eq. (\ref{eq:Sc}), dove $\hat{H}$ rappresenta l'operatore Hamiltoniano \footnote{Si ricorda che le grandezze riportate sotto il simbolo $\, \hat{ } \,$ rapprenatano degli operatori lineari in uno spazio di Hilbert}
\begin{equation}
    \centering
    \hat{H} (\vec{x}, t) = \frac{1}{2m} (\hat{p} - q \hat{A}(\vec{x}, t) )^2 + \hat{V}(\vec{x}, t) - g \frac{q}{2m} \hat{S} \cdot \hat{B}(\vec{x},t) \; \text{.}
    \label{eq:H}
\end{equation}
Di seguito, nei casi presi in esame, si utilizza una forma semplificata dell'eq. (\ref{eq:Sc}) e dell'operatore in eq. (\ref{eq:H}), poiché verranno utilizzati nella loro forma monodimensionale e senza considerare interazioni di natura elettromagnetica. Questa semplificazione non modifica il formalismo matematico utilizzato, ma permette di rendere più esplicative e chiare le espressioni che seguiranno; è dunque possibile riadattare tali espressioni e generalizzarle utilizzanado lo stesso procedimento. L'equazione risolta nelle simulazioni risulta essere
\begin{equation}
    \centering
    i \hbar \frac{\partial}{\partial t} \psi (x, t) =  \hat{H} (\vec{x}, t) \psi (\vec{x}, t)   \qquad \text{con} \quad \hat{H} (\vec{x}, t) = \frac{\hat{p}^2}{2m}  + \hat{V}(\vec{x}, t) = \hat{T} + \hat{V} \; \text{,}
    \label{eq:Sc_1D}
\end{equation}
dove si ricorda che, nella rappresentazione degli $x$, 
\begin{equation}
    \centering
    \hat{p} = - i \hbar \frac{\partial } {\partial x} \; \text{.}
    \label{eq:p}
\end{equation}
La soluzione formale dell'equaizone di Schr\"odinger fa uso dell'operatore unitario di evoluzione temporale 
\begin{equation}
    \centering
    \psi (x,t) = \hat{U}(t, t_{0}) \, \psi (x, t_0) \quad \text{con} \quad \hat{U}(t, t_{0}) = T  \exp \left(\frac{-i}{\hbar} \int _{t_{0}} ^{t} dt' \; \hat{H}(x,t') \right) \; \text{.}
    \label{eq:U}
\end{equation}
% Devo paarlare del motivo per cui viene semplicata
Tale soluzione, non essendo esplicitamente analitica, non può essere risolta in maniera esatta se non in casi molto particolari.
Se però si considera l'espressione infinititesima dell'operatore $\hat{U}$, risulta
\begin{equation}
    \centering
    \hat{U}(t + \delta t, t) = \exp \left(\frac{-i}{\hbar} {H}(x,t) \delta t \right)  \; \text{,}
    \label{eq:U_inf}
\end{equation}
che può essere applicato senza difficoltà alla funzione d'onda $\psi(x,t)$ il numero di volte necessarie a ottenere lo stesso risultato dell'eq. (\ref{eq:U}).
L'unica difficoltà risiede nell'espansione dell'operatore, poiché in forma esplicita  
\begin{equation}
    \centering
    \hat{U}(t + \delta t, t) =  \exp \left(\frac{-i}{\hbar} \delta t (\hat{V} + \hat{T}) \right) \; \text{,}
    \label{eq:U_esplicit} 
\end{equation}
dove $\hat{V}$ e $\hat{T}$ sono due operatori che non commutano. Questo rende l'espansione dell'operatore $\hat{U}$ non banale.