\chapter*{Conclusioni}
\label{ch:conclusioni}

Grazie al lavoro esposto in sezione (\ref{ch:Validazione}) è possibile affermare che la soluzione numerica
%cambia
generata tramite la simulazione è  del tutto compatibile con la soluzione esatta. 
È quindi possibile utilizzare questo strumento per risolvere un qualsiasi problema monodimensionale e ottenere risultati concreti come in sezione (\ref{ch:applicazioni}). 

%amplia
Il formalismo esposto in sezione (\ref{ch:teoria}) è dunque verificatto e applicato. Tale formalismo è generalizzabile per qualsiasi operatore $\hat{H}$ in un numero arbitario di dimensioni, quindi si può estendere ad ogni sistema quantistico. I problemi che si possono riscontrare nella generalizzazione, oltre al fatto che la comprensione delle formule matematica diviene più difficoltosa, riguardano soprattutto il costo computazionale e le strategie di visualizzazione dei risultati. Ogni volta che si aumentano le dimensioni o i gradi di libertà del sistema (ad esempio considerando lo spin), è necessario ampliare la grandezza dei vettori che contengono la funzione d'onda. Nello spazio delle coordinate l'aumento di punti da considerare scala con $N^d$, dove di rappresenta le dimensioni. Inoltre per rappresentare funzioni d'onda multidimensionali è necessario animare uno spazio tridimesionale o quadridimensionale, il che anche se possibile rende i risultati più difficili da interpretare. 

